\documentclass[11pt,a4paper]{article}
\usepackage[utf8]{inputenc}
\usepackage[T1]{fontenc}
\usepackage{geometry}
\usepackage{graphicx}
\usepackage{hyperref}
\usepackage{listings}
\usepackage{xcolor}
\usepackage{booktabs}
\usepackage{longtable}
\usepackage{enumitem}
\usepackage{titlesec}
\usepackage{fancyhdr}
\usepackage{amsmath}

\geometry{margin=1in}
\hypersetup{
    colorlinks=true,
    linkcolor=blue,
    filecolor=magenta,      
    urlcolor=cyan,
    pdftitle={Mortgage Platform AWS Deployment Documentation},
    pdfauthor={Kwalifai Platform},
    pdfsubject={AWS Deployment Strategy and Cost Analysis}
}

\pagestyle{fancy}
\fancyhf{}
\fancyhead[L]{Mortgage Platform - AWS Deployment}
\fancyhead[R]{\thepage}
\fancyfoot[C]{Confidential - Kwalifai Platform}

\title{\textbf{Mortgage Platform: Features, AWS Deployment Strategy, and Cost Analysis}}
\author{Kwalifai Platform Development Team}
\date{\today}

\lstset{
    basicstyle=\ttfamily\small,
    breaklines=true,
    frame=single,
    backgroundcolor=\color{gray!10},
    language=bash
}

\begin{document}

\maketitle
\thispagestyle{fancy}

\tableofcontents
\newpage

\section{Executive Summary}

This document provides a comprehensive overview of the Kwalifai Mortgage Platform, detailing all features, AWS deployment architecture, service selection rationale, and cost estimates. The platform is a microservices-based mortgage solution offering calculators, rate alerts, statement analysis, and AI-powered mortgage assistance.

\section{Platform Overview}

\subsection{Architecture}

The platform follows a microservices architecture pattern with the following components:

\begin{itemize}
    \item \textbf{Frontend}: Next.js 14 application with App Router, TypeScript, and Tailwind CSS
    \item \textbf{API Gateway}: Single entry point routing requests to backend services
    \item \textbf{Auth Service}: User authentication and authorization with JWT tokens
    \item \textbf{Calculator Service}: Mortgage payment calculations and scenario management
    \item \textbf{Rate Alert Service}: Interest rate monitoring and notification system
    \item \textbf{Statement Analysis Service}: Mortgage statement upload, storage, and analysis
    \item \textbf{AI Service (LOWI)}: AI-powered mortgage assistant with OpenAI integration
    \item \textbf{Database}: PostgreSQL with Prisma ORM
    \item \textbf{Shared Packages}: Common TypeScript types, utilities, and constants
\end{itemize}

\subsection{Technology Stack}

\begin{itemize}
    \item \textbf{Frontend}: Next.js 14, React 18, TypeScript, Tailwind CSS
    \item \textbf{Backend}: Node.js, Express.js, TypeScript
    \item \textbf{Database}: PostgreSQL with Prisma ORM
    \item \textbf{Package Manager}: pnpm workspaces
    \item \textbf{AI Integration}: OpenAI GPT-4 (via AI Service)
    \item \textbf{Email}: SendGrid (planned) / AWS SES (recommended)
\end{itemize}

\section{Feature Documentation}

\subsection{1. Smart Mortgage Calculators}

\subsubsection{Features}
\begin{itemize}
    \item \textbf{Mortgage Calculator}: Calculate monthly payments, total interest, and amortization schedules
    \item \textbf{Affordability Calculator}: Determine maximum affordable loan amount based on income
    \item \textbf{Refinance Calculator}: Compare current mortgage with refinancing options
    \item \textbf{Amortization Calculator}: Generate detailed payment schedules over loan term
    \item \textbf{Scenario Saving}: Save and retrieve calculation scenarios for authenticated users
    \item \textbf{Multi-Scenario Comparison}: Compare up to four loan options side-by-side
\end{itemize}

\subsubsection{Technical Implementation}
\begin{itemize}
    \item Service: \texttt{calculator-service} (Port 3003)
    \item Endpoints:
    \begin{itemize}
        \item \texttt{POST /api/v1/calculator/mortgage}
        \item \texttt{POST /api/v1/calculator/affordability}
        \item \texttt{POST /api/v1/calculator/scenarios}
        \item \texttt{GET /api/v1/calculator/scenarios}
        \item \texttt{GET /api/v1/calculator/scenarios/:id}
        \item \texttt{DELETE /api/v1/calculator/scenarios/:id}
    \end{itemize}
    \item Rate Limiting: 50 requests per 15 minutes per IP
    \item Database: \texttt{CalculatorScenario} model stores user scenarios
\end{itemize}

\subsection{2. Rate Alerts System}

\subsubsection{Features}
\begin{itemize}
    \item \textbf{Target Rate Monitoring}: Set target interest rates for specific loan types
    \item \textbf{Automated Notifications}: Receive alerts when rates drop to target threshold
    \item \textbf{Multiple Loan Types}: Support for 30-Year Fixed, 15-Year Fixed, FHA, VA, USDA, Jumbo, ARM
    \item \textbf{Lead Generation}: Automatic lead creation with high lead score (85) when alerts are created
    \item \textbf{Alert Management}: Create, view, update, and deactivate rate alerts
    \item \textbf{Idempotency}: Safe retry mechanism using idempotency keys
    \item \textbf{Multi-Tenant Support}: Broker and loan officer context tracking
\end{itemize}

\subsubsection{Technical Implementation}
\begin{itemize}
    \item Service: \texttt{rate-alert-service} (Port 3004)
    \item Endpoints:
    \begin{itemize}
        \item \texttt{POST /api/v1/rate-alerts}
        \item \texttt{GET /api/v1/rate-alerts}
        \item \texttt{GET /api/v1/rate-alerts/:id}
        \item \texttt{PUT /api/v1/rate-alerts/:id}
        \item \texttt{DELETE /api/v1/rate-alerts/:id}
    \end{itemize}
    \item Database: \texttt{RateAlert} and \texttt{RateAlertNotification} models
    \item Background Jobs: Rate monitoring service checks rates periodically
    \item Notification Channels: Email (SendGrid/SES), SMS (future), In-app (future)
\end{itemize}

\subsection{3. Statement Analysis Service}

\subsubsection{Features}
\begin{itemize}
    \item \textbf{Statement Upload}: Accept PDF/image uploads via multipart form data
    \item \textbf{Manual Data Entry}: Form-based input for statement details
    \item \textbf{Refinance Analysis}: Calculate potential savings and break-even points
    \item \textbf{Market Comparison}: Compare current rate with market rates
    \item \textbf{Recommendations}: AI-powered refinancing recommendations
    \item \textbf{Versioned Analysis}: Support for V1 (manual), V2 (OCR - planned), V3 (AI - planned)
    \item \textbf{File Storage}: Local filesystem (current), S3/GCS ready (future)
    \item \textbf{Session Support}: Track anonymous user sessions
\end{itemize}

\subsubsection{Technical Implementation}
\begin{itemize}
    \item Service: \texttt{statement-analysis-service} (Port 3005)
    \item Endpoints:
    \begin{itemize}
        \item \texttt{POST /api/upload-statement}
        \item \texttt{GET /api/statement/:id}
    \end{itemize}
    \item Database: \texttt{Statement}, \texttt{StatementFile}, \texttt{StatementAnalysis} models
    \item File Storage: Abstracted provider interface (Local/S3/GCS)
    \item Analysis Engine: Versioned architecture for future OCR and AI integration
    \item Rate Limiting: 20 uploads per 15 minutes per IP
\end{itemize}

\subsection{4. AI-Powered Mortgage Assistant (LOWI)}

\subsubsection{Features}
\begin{itemize}
    \item \textbf{Natural Language Chat}: Conversational interface for mortgage questions
    \item \textbf{Intent Classification}: Understands user queries and routes to appropriate tools
    \item \textbf{Integrated Tools}: 
    \begin{itemize}
        \item Run mortgage calculators
        \item Get current mortgage rates
        \item Create rate alerts
        \item Analyze statements
        \item Get user profile information
        \item List user's rate alerts
        \item Save calculations
    \end{itemize}
    \item \textbf{Guardrails}: Content filtering, response validation, audit logging
    \item \textbf{Fallback System}: Graceful degradation when AI service is unavailable
    \item \textbf{Session Management}: Maintains conversation context
    \item \textbf{Correlation Tracking}: Request tracing for debugging
\end{itemize}

\subsubsection{Technical Implementation}
\begin{itemize}
    \item Service: \texttt{ai-service} (Port 3006)
    \item Endpoints:
    \begin{itemize}
        \item \texttt{POST /api/v1/chat}
        \item \texttt{GET /api/v1/chat} (health check)
    \end{itemize}
    \item AI Provider: OpenAI GPT-4 (configurable model)
    \item Tools: Function calling with OpenAI
    \item Middleware: Rate limiting, authentication, correlation ID, audit logging
    \item Scheduler: Background task processing for async operations
\end{itemize}

\subsection{5. Authentication and Authorization}

\subsubsection{Features}
\begin{itemize}
    \item \textbf{User Registration}: Email and phone-based registration
    \item \textbf{Email Verification}: OTP-based email verification
    \item \textbf{Phone Verification}: OTP-based phone verification
    \item \textbf{Password Reset}: Secure password reset flow
    \item \textbf{JWT Tokens}: Access and refresh token mechanism
    \item \textbf{Session Management}: Refresh token rotation and revocation
    \item \textbf{Multi-Factor Authentication}: Support for MFA (future)
    \item \textbf{User Status Management}: PENDING, ACTIVE, INACTIVE, SUSPENDED, DELETED
\end{itemize}

\subsubsection{Technical Implementation}
\begin{itemize}
    \item Service: \texttt{auth-service} (Port 3002)
    \item Endpoints:
    \begin{itemize}
        \item \texttt{POST /api/v1/auth/register}
        \item \texttt{POST /api/v1/auth/verify-email}
        \item \texttt{POST /api/v1/auth/verify-phone}
        \item \texttt{POST /api/v1/auth/login}
        \item \texttt{POST /api/v1/auth/refresh}
        \item \texttt{POST /api/v1/auth/logout}
    \end{itemize}
    \item Security: bcrypt password hashing, JWT signing, secure token storage
    \item Database: \texttt{User}, \texttt{VerificationCode}, \texttt{RefreshToken} models
\end{itemize}

\subsection{6. API Gateway}

\subsubsection{Features}
\begin{itemize}
    \item \textbf{Single Entry Point}: All API requests routed through gateway
    \item \textbf{Request Routing}: Intelligent routing to appropriate microservices
    \item \textbf{Correlation ID}: Request tracing across services
    \item \textbf{Rate Limiting}: Global rate limiting (100 req/15min per IP)
    \item \textbf{CORS Management}: Configurable CORS policies
    \item \textbf{Swagger Documentation}: Interactive API documentation
    \item \textbf{Health Checks}: Service health and readiness monitoring
    \item \textbf{Tenant Resolution}: Multi-tenant support with broker/loan officer context
\end{itemize}

\subsubsection{Technical Implementation}
\begin{itemize}
    \item Service: \texttt{api-gateway} (Port 3001)
    \item Technology: Express.js with http-proxy-middleware
    \item Middleware: Correlation ID, rate limiting, tenant resolution, proxy
    \item Documentation: Swagger/OpenAPI 3.0
\end{itemize}

\section{AWS Deployment Architecture}

\subsection{Recommended AWS Services}

\subsubsection{Compute Services}

\paragraph{Amazon ECS with Fargate}
\begin{itemize}
    \item \textbf{Purpose}: Container orchestration for microservices
    \item \textbf{Why ECS Fargate}:
    \begin{itemize}
        \item Serverless container management (no EC2 management)
        \item Automatic scaling based on demand
        \item Cost-effective for variable workloads
        \item Integrated with other AWS services
        \item Simpler than EKS for this use case
    \end{itemize}
    \item \textbf{Why Not EKS}: 
    \begin{itemize}
        \item Overkill for current scale
        \item Higher operational complexity
        \item More expensive (control plane costs)
        \item Requires Kubernetes expertise
    \end{itemize}
    \item \textbf{Why Not EC2 Direct}:
    \begin{itemize}
        \item Manual scaling and management overhead
        \item Less efficient resource utilization
        \item Requires infrastructure management
    \end{itemize}
    \item \textbf{Why Not Lambda}:
    \begin{itemize}
        \item Cold start latency for some services
        \item 15-minute execution limit
        \item Not ideal for long-running connections (WebSockets, if needed)
        \item Container services better for consistent workloads
    \end{itemize}
\end{itemize}

\paragraph{Application Load Balancer (ALB)}
\begin{itemize}
    \item \textbf{Purpose}: Distribute traffic across services
    \item \textbf{Why ALB}:
    \begin{itemize}
        \item Layer 7 routing (path-based, host-based)
        \item SSL/TLS termination
        \item Health checks and auto-scaling integration
        \item WAF integration for security
    \end{itemize}
    \item \textbf{Why Not Classic Load Balancer}:
    \begin{itemize}
        \item Legacy service, less features
        \item No path-based routing
    \end{itemize}
    \item \textbf{Why Not NLB}:
    \begin{itemize}
        \item Layer 4 only, no HTTP/HTTPS features
        \item Less suitable for API routing
    \end{itemize}
\end{itemize}

\subsubsection{Database Services}

\paragraph{Amazon RDS PostgreSQL}
\begin{itemize}
    \item \textbf{Purpose}: Managed PostgreSQL database
    \item \textbf{Why RDS}:
    \begin{itemize}
        \item Automated backups and point-in-time recovery
        \item Automatic patching and updates
        \item Multi-AZ for high availability
        \item Read replicas for scaling reads
        \item Encryption at rest and in transit
        \item Managed maintenance windows
    \end{itemize}
    \item \textbf{Why Not Aurora}:
    \begin{itemize}
        \item Higher cost (2x-3x more expensive)
        \item Overkill for initial scale
        \item Can migrate later if needed
    \end{itemize}
    \item \textbf{Why Not Self-Managed EC2}:
    \begin{itemize}
        \item Operational overhead (backups, patching, monitoring)
        \item Higher risk of downtime
        \item Requires database administration expertise
    \end{itemize}
    \item \textbf{Why Not DynamoDB}:
    \begin{itemize}
        \item NoSQL doesn't fit relational data model
        \item Prisma ORM requires SQL database
        \item Complex queries and joins needed
    \end{itemize}
\end{itemize}

\subsubsection{Storage Services}

\paragraph{Amazon S3}
\begin{itemize}
    \item \textbf{Purpose}: File storage for mortgage statements
    \item \textbf{Why S3}:
    \begin{itemize}
        \item 99.999999999\% (11 9's) durability
        \item Virtually unlimited scalability
        \item Cost-effective storage tiers
        \item Lifecycle policies for archival
        \item Versioning and encryption
        \item CDN integration (CloudFront)
    \end{itemize}
    \item \textbf{Why Not EBS}:
    \begin{itemize}
        \item Attached to single EC2 instance
        \item Not suitable for shared file storage
        \item Limited scalability
    \end{itemize}
    \item \textbf{Why Not EFS}:
    \begin{itemize}
        \item More expensive than S3
        \item Overkill for file uploads
        \item Better for shared filesystem needs
    \end{itemize}
    \item \textbf{Why Not Glacier}:
    \begin{itemize}
        \item Too slow for active file access
        \item Better for long-term archival
        \item Can use S3 lifecycle to Glacier
    \end{itemize}
\end{itemize}

\paragraph{Amazon CloudFront}
\begin{itemize}
    \item \textbf{Purpose}: CDN for frontend and static assets
    \item \textbf{Why CloudFront}:
    \begin{itemize}
        \item Global edge locations for low latency
        \item SSL/TLS termination
        \item DDoS protection
        \item Cost-effective data transfer
        \item Integration with S3 and ALB
    \end{itemize}
    \item \textbf{Why Not Other CDNs}:
    \begin{itemize}
        \item Better AWS integration
        \item Unified billing and management
        \item Competitive pricing
    \end{itemize}
\end{itemize}

\subsubsection{Networking Services}

\paragraph{Amazon VPC}
\begin{itemize}
    \item \textbf{Purpose}: Isolated network environment
    \item \textbf{Configuration}:
    \begin{itemize}
        \item Public subnets for ALB
        \item Private subnets for ECS tasks
        \item Private subnets for RDS
        \item NAT Gateway for outbound internet access
    \end{itemize}
\end{itemize}

\paragraph{Amazon Route 53}
\begin{itemize}
    \item \textbf{Purpose}: DNS management
    \item \textbf{Why Route 53}:
    \begin{itemize}
        \item Integrated with AWS services
        \item Health checks and failover
        \item Low latency DNS resolution
    \end{itemize}
\end{itemize}

\subsubsection{Security Services}

\paragraph{AWS Secrets Manager}
\begin{itemize}
    \item \textbf{Purpose}: Store sensitive configuration (API keys, DB passwords)
    \item \textbf{Why Secrets Manager}:
    \begin{itemize}
        \item Automatic rotation support
        \item Fine-grained access control
        \item Audit logging
        \item Integration with ECS
    \end{itemize}
    \item \textbf{Why Not Parameter Store}:
    \begin{itemize}
        \item Secrets Manager better for secrets (encryption, rotation)
        \item Parameter Store better for non-sensitive config
    \end{itemize}
\end{itemize}

\paragraph{AWS WAF}
\begin{itemize}
    \item \textbf{Purpose}: Web application firewall
    \item \textbf{Why WAF}:
    \begin{itemize}
        \item Protection against common attacks (SQL injection, XSS)
        \item Rate-based rules
        \item Geographic restrictions
        \item Integration with ALB
    \end{itemize}
\end{itemize}

\subsubsection{Monitoring and Logging}

\paragraph{Amazon CloudWatch}
\begin{itemize}
    \item \textbf{Purpose}: Monitoring, logging, and alerting
    \item \textbf{Why CloudWatch}:
    \begin{itemize}
        \item Native AWS integration
        \item Log aggregation from all services
        \item Metrics and dashboards
        \item Alarms and notifications
        \item Cost-effective for AWS-native monitoring
    \end{itemize}
    \item \textbf{Why Not Datadog/New Relic}:
    \begin{itemize}
        \item Higher cost (can add later if needed)
        \item CloudWatch sufficient for initial deployment
        \item Can integrate third-party tools later
    \end{itemize}
\end{itemize}

\subsubsection{Email Services}

\paragraph{Amazon SES}
\begin{itemize}
    \item \textbf{Purpose}: Transactional emails (verification, notifications)
    \item \textbf{Why SES}:
    \begin{itemize}
        \item Cost-effective (\$0.10 per 1,000 emails)
        \item High deliverability
        \item Integration with other AWS services
        \item Better than SendGrid for AWS-native stack
    \end{itemize}
    \item \textbf{Why Not SendGrid}:
    \begin{itemize}
        \item Additional external dependency
        \item Less integrated with AWS ecosystem
        \item Similar pricing, but SES is more integrated
    \end{itemize}
\end{itemize}

\subsubsection{Background Jobs}

\paragraph{AWS Lambda}
\begin{itemize}
    \item \textbf{Purpose}: Rate monitoring and scheduled tasks
    \item \textbf{Why Lambda}:
    \begin{itemize}
        \item Serverless, pay-per-execution
        \item EventBridge integration for scheduling
        \item Cost-effective for periodic tasks
        \item No infrastructure management
    \end{itemize}
    \item \textbf{Why Not ECS Scheduled Tasks}:
    \begin{itemize}
        \item Lambda simpler for simple scheduled jobs
        \item Lower cost for infrequent tasks
        \item Can use ECS if tasks become complex
    \end{itemize}
\end{itemize}

\paragraph{Amazon EventBridge}
\begin{itemize}
    \item \textbf{Purpose}: Schedule rate monitoring jobs
    \item \textbf{Why EventBridge}:
    \begin{itemize}
        \item Serverless event scheduling
        \item Cron-like scheduling
        \item Integration with Lambda
    \end{itemize}
\end{itemize}

\subsection{Deployment Architecture Diagram}

\begin{verbatim}
┌─────────────────────────────────────────────────────────────┐
│                    CloudFront (CDN)                          │
│              (Frontend Static Assets)                        │
└────────────────────┬────────────────────────────────────────┘
                     │
                     ▼
┌─────────────────────────────────────────────────────────────┐
│              Application Load Balancer (ALB)                 │
│              + AWS WAF (Security)                             │
└──────┬───────────────────────────────┬───────────────────────┘
       │                               │
       ▼                               ▼
┌──────────────────┐         ┌──────────────────────┐
│   ECS Fargate    │         │    ECS Fargate       │
│   (Frontend)     │         │  (API Gateway)       │
│   Next.js App    │         │  Express Service     │
└──────────────────┘         └──────┬───────────────┘
                                    │
                    ┌───────────────┼───────────────┐
                    │               │               │
                    ▼               ▼               ▼
         ┌──────────────┐  ┌──────────────┐  ┌──────────────┐
         │ ECS Fargate  │  │ ECS Fargate  │  │ ECS Fargate  │
         │ Auth Service │  │ Calculator   │  │ Rate Alert   │
         └──────────────┘  │ Service      │  │ Service      │
                           └──────────────┘  └──────────────┘
                                    │
                    ┌───────────────┼───────────────┐
                    │               │               │
                    ▼               ▼               ▼
         ┌──────────────┐  ┌──────────────┐  ┌──────────────┐
         │ ECS Fargate  │  │ ECS Fargate  │  │   Lambda     │
         │ Statement    │  │ AI Service   │  │ Rate Monitor │
         │ Analysis     │  │ (LOWI)       │  │ (Scheduled)   │
         └──────────────┘  └──────────────┘  └──────────────┘
                    │
                    ▼
         ┌──────────────────────┐
         │   Amazon RDS          │
         │   PostgreSQL          │
         │   (Multi-AZ)          │
         └──────────────────────┘
                    │
                    ▼
         ┌──────────────────────┐
         │   Amazon S3          │
         │   (File Storage)     │
         └──────────────────────┘
                    │
                    ▼
         ┌──────────────────────┐
         │   Amazon SES         │
         │   (Email Service)    │
         └──────────────────────┘
\end{verbatim}

\section{Deployment Steps}

\subsection{Prerequisites}
\begin{enumerate}
    \item AWS Account with appropriate IAM permissions
    \item AWS CLI configured
    \item Docker installed locally
    \item Domain name (optional, for Route 53)
\end{enumerate}

\subsection{Step 1: Create VPC and Networking}
\begin{lstlisting}
# Create VPC with public and private subnets
# Configure Internet Gateway and NAT Gateway
# Set up security groups for services
\end{lstlisting}

\subsection{Step 2: Set Up RDS PostgreSQL}
\begin{lstlisting}
# Create RDS PostgreSQL instance
# Configure Multi-AZ for high availability
# Set up automated backups
# Create database and run Prisma migrations
\end{lstlisting}

\subsection{Step 3: Create S3 Bucket}
\begin{lstlisting}
# Create S3 bucket for file storage
# Configure bucket policies and CORS
# Set up lifecycle policies for cost optimization
\end{lstlisting}

\subsection{Step 4: Set Up ECR (Elastic Container Registry)}
\begin{lstlisting}
# Create ECR repositories for each service
# Build and push Docker images
\end{lstlisting}

\subsection{Step 5: Create ECS Cluster and Services}
\begin{lstlisting}
# Create ECS cluster
# Create task definitions for each service
# Configure auto-scaling policies
# Set up service discovery (optional)
\end{lstlisting}

\subsection{Step 6: Configure Application Load Balancer}
\begin{lstlisting}
# Create ALB in public subnets
# Configure target groups for each service
# Set up path-based routing rules
# Configure SSL/TLS certificates (ACM)
\end{lstlisting}

\subsection{Step 7: Set Up CloudFront}
\begin{lstlisting}
# Create CloudFront distribution
# Point to ALB origin
# Configure caching policies
# Set up custom domain (optional)
\end{lstlisting}

\subsection{Step 8: Configure Secrets Manager}
\begin{lstlisting}
# Store database credentials
# Store API keys (OpenAI, etc.)
# Store JWT secrets
# Configure IAM roles for ECS tasks
\end{lstlisting}

\subsection{Step 9: Set Up Monitoring}
\begin{lstlisting}
# Configure CloudWatch log groups
# Set up CloudWatch alarms
# Create dashboards for monitoring
# Configure SNS for alerts
\end{lstlisting}

\subsection{Step 10: Deploy Lambda Functions}
\begin{lstlisting}
# Create Lambda function for rate monitoring
# Configure EventBridge schedule (hourly)
# Set up IAM roles and permissions
\end{lstlisting}

\section{Cost Analysis}

\subsection{Monthly Cost Estimates (US East Region)}

\subsubsection{Compute Costs}

\begin{table}[h]
\centering
\begin{tabular}{lrrr}
\toprule
\textbf{Service} & \textbf{Configuration} & \textbf{Units} & \textbf{Cost/Month} \\
\midrule
ECS Fargate (Frontend) & 0.5 vCPU, 1GB RAM & 1 task & \$15 \\
ECS Fargate (API Gateway) & 0.25 vCPU, 0.5GB RAM & 1 task & \$8 \\
ECS Fargate (Auth Service) & 0.25 vCPU, 0.5GB RAM & 1 task & \$8 \\
ECS Fargate (Calculator) & 0.25 vCPU, 0.5GB RAM & 1 task & \$8 \\
ECS Fargate (Rate Alert) & 0.25 vCPU, 0.5GB RAM & 1 task & \$8 \\
ECS Fargate (Statement) & 0.5 vCPU, 1GB RAM & 1 task & \$15 \\
ECS Fargate (AI Service) & 0.5 vCPU, 1GB RAM & 1 task & \$15 \\
Application Load Balancer & Standard & 1 ALB & \$22 \\
\bottomrule
\textbf{Total Compute} & & & \textbf{\$99} \\
\bottomrule
\end{tabular}
\caption{Compute Service Costs}
\end{table}

\subsubsection{Database Costs}

\begin{table}[h]
\centering
\begin{tabular}{lrrr}
\toprule
\textbf{Service} & \textbf{Configuration} & \textbf{Units} & \textbf{Cost/Month} \\
\midrule
RDS PostgreSQL (db.t3.medium) & 2 vCPU, 4GB RAM, 100GB & 1 instance & \$75 \\
RDS Storage & 100GB GP3 & 100 GB & \$11 \\
RDS Backup Storage & 100GB & 100 GB & \$11 \\
\bottomrule
\textbf{Total Database} & & & \textbf{\$97} \\
\bottomrule
\end{tabular}
\caption{Database Service Costs}
\end{table}

\subsubsection{Storage and CDN Costs}

\begin{table}[h]
\centering
\begin{tabular}{lrrr}
\toprule
\textbf{Service} & \textbf{Configuration} & \textbf{Usage} & \textbf{Cost/Month} \\
\midrule
S3 Standard Storage & First 50TB & 10 GB & \$0.23 \\
S3 PUT Requests & & 10,000 requests & \$0.05 \\
S3 GET Requests & & 100,000 requests & \$0.40 \\
CloudFront Data Transfer & & 100 GB & \$8.50 \\
CloudFront Requests & & 1M requests & \$0.75 \\
\bottomrule
\textbf{Total Storage/CDN} & & & \textbf{\$9.93} \\
\bottomrule
\end{tabular}
\caption{Storage and CDN Costs}
\end{table}

\subsubsection{Networking Costs}

\begin{table}[h]
\centering
\begin{tabular}{lrrr}
\toprule
\textbf{Service} & \textbf{Configuration} & \textbf{Usage} & \textbf{Cost/Month} \\
\midrule
NAT Gateway & Standard & 1 gateway & \$32.40 \\
NAT Gateway Data Transfer & & 100 GB & \$4.50 \\
Route 53 Hosted Zone & & 1 zone & \$0.50 \\
Route 53 Queries & & 1M queries & \$0.40 \\
VPC (Free) & & & \$0 \\
\bottomrule
\textbf{Total Networking} & & & \textbf{\$37.80} \\
\bottomrule
\end{tabular}
\caption{Networking Costs}
\end{table}

\subsubsection{Security and Management Costs}

\begin{table}[h]
\centering
\begin{tabular}{lrrr}
\toprule
\textbf{Service} & \textbf{Configuration} & \textbf{Usage} & \textbf{Cost/Month} \\
\midrule
Secrets Manager & & 7 secrets & \$3.50 \\
AWS WAF & Web ACL + Rules & 1 ACL & \$5 \\
CloudWatch Logs & & 10 GB & \$5 \\
CloudWatch Metrics & & 150 metrics & \$1.50 \\
CloudWatch Alarms & & 10 alarms & \$0 \\
\bottomrule
\textbf{Total Security/Management} & & & \textbf{\$15} \\
\bottomrule
\end{tabular}
\caption{Security and Management Costs}
\end{table}

\subsubsection{Additional Services}

\begin{table}[h]
\centering
\begin{tabular}{lrrr}
\toprule
\textbf{Service} & \textbf{Configuration} & \textbf{Usage} & \textbf{Cost/Month} \\
\midrule
Amazon SES & & 10,000 emails & \$1 \\
Lambda (Rate Monitor) & & 720 invocations & \$0 \\
EventBridge & & 720 rules & \$0 \\
ECR Storage & & 5 GB & \$0.50 \\
\bottomrule
\textbf{Total Additional} & & & \textbf{\$1.50} \\
\bottomrule
\end{tabular}
\caption{Additional Service Costs}
\end{table}

\subsection{Total Monthly Cost Summary}

\begin{table}[h]
\centering
\begin{tabular}{lr}
\toprule
\textbf{Category} & \textbf{Cost/Month} \\
\midrule
Compute (ECS + ALB) & \$99.00 \\
Database (RDS) & \$97.00 \\
Storage \& CDN (S3 + CloudFront) & \$9.93 \\
Networking (NAT + Route 53) & \$37.80 \\
Security \& Management & \$15.00 \\
Additional Services & \$1.50 \\
\midrule
\textbf{Total Estimated Monthly Cost} & \textbf{\$260.23} \\
\bottomrule
\end{tabular}
\caption{Total Monthly Cost Breakdown}
\end{table}

\subsection{Cost Optimization Recommendations}

\begin{enumerate}
    \item \textbf{Reserved Instances}: Consider RDS Reserved Instances for 1-3 year commitment (30-40\% savings)
    \item \textbf{S3 Lifecycle Policies}: Move old files to S3 Glacier for archival (80\% savings)
    \item \textbf{Auto-Scaling}: Scale down services during low-traffic periods
    \item \textbf{CloudFront Caching}: Optimize cache policies to reduce origin requests
    \item \textbf{NAT Gateway}: Consider NAT Instance for development (lower cost, less reliable)
    \item \textbf{Spot Instances}: Not applicable for Fargate, but consider for future EC2-based services
    \item \textbf{Monitoring}: Review CloudWatch log retention (reduce from 30 days if not needed)
\end{enumerate}

\subsection{Scaling Cost Projections}

\begin{table}[h]
\centering
\begin{tabular}{lrrr}
\toprule
\textbf{Traffic Level} & \textbf{Monthly Users} & \textbf{Estimated Cost} & \textbf{Notes} \\
\midrule
Low (Startup) & 1,000 & \$260 & Current estimate \\
Medium & 10,000 & \$450 & +ECS tasks, +RDS size \\
High & 100,000 & \$1,200 & +Read replicas, +ALB scaling \\
Enterprise & 1,000,000+ & \$3,500+ & Multi-region, Aurora \\
\bottomrule
\end{tabular}
\caption{Cost Projections by Scale}
\end{table}

\section{Why These AWS Services?}

\subsection{Service Selection Rationale}

\subsubsection{Why ECS Fargate Over Alternatives}
\begin{itemize}
    \item \textbf{vs. EC2}: No server management, automatic scaling, better resource utilization
    \item \textbf{vs. EKS}: Lower complexity, lower cost, sufficient for current needs
    \item \textbf{vs. Lambda}: Better for long-running services, WebSocket support, consistent performance
    \item \textbf{vs. Elastic Beanstalk}: More control, better for microservices, container-native
\end{itemize}

\subsubsection{Why RDS Over Alternatives}
\begin{itemize}
    \item \textbf{vs. Aurora}: Lower cost, sufficient performance, can migrate later
    \item \textbf{vs. Self-Managed}: Less operational overhead, automated backups, managed updates
    \item \textbf{vs. DynamoDB}: Relational data model, Prisma ORM compatibility, complex queries
\end{itemize}

\subsubsection{Why S3 Over Alternatives}
\begin{itemize}
    \item \textbf{vs. EBS}: Shared access, unlimited scale, better for file uploads
    \item \textbf{vs. EFS}: Lower cost, better for object storage, lifecycle policies
    \item \textbf{vs. Local Storage}: Durability, scalability, backup built-in
\end{itemize}

\subsubsection{Why CloudFront Over Alternatives}
\begin{itemize}
    \item \textbf{vs. Other CDNs}: Better AWS integration, unified billing, competitive pricing
    \item \textbf{vs. No CDN}: Lower latency globally, DDoS protection, cost-effective data transfer
\end{itemize}

\section{Security Considerations}

\subsection{Network Security}
\begin{itemize}
    \item VPC with private subnets for services
    \item Security groups with least-privilege access
    \item NAT Gateway for controlled outbound access
    \item No direct internet access to services
\end{itemize}

\subsection{Data Security}
\begin{itemize}
    \item Encryption at rest (RDS, S3, EBS)
    \item Encryption in transit (TLS/SSL)
    \item Secrets Manager for sensitive data
    \item IAM roles with least privilege
\end{itemize}

\subsection{Application Security}
\begin{itemize}
    \item AWS WAF for DDoS and attack protection
    \item Rate limiting at API Gateway
    \item Input validation in all services
    \item Audit logging for compliance
\end{itemize}

\section{Monitoring and Observability}

\subsection{CloudWatch Metrics}
\begin{itemize}
    \item ECS task CPU and memory utilization
    \item ALB request counts and latency
    \item RDS connection counts and query performance
    \item Lambda invocation metrics
    \item Custom application metrics
\end{itemize}

\subsection{CloudWatch Logs}
\begin{itemize}
    \item Centralized logging from all services
    \item Log retention policies
    \item Log insights for querying
    \item Correlation ID tracking
\end{itemize}

\subsection{Alarms}
\begin{itemize}
    \item High CPU/memory utilization
    \item Database connection issues
    \item High error rates
    \item Service unavailability
    \item Cost threshold alerts
\end{itemize}

\section{Disaster Recovery and Backup}

\subsection{Database Backups}
\begin{itemize}
    \item Automated daily backups (RDS)
    \item 7-day retention (configurable)
    \item Point-in-time recovery
    \item Multi-AZ for high availability
\end{itemize}

\subsection{File Storage}
\begin{itemize}
    \item S3 versioning (optional)
    \item Cross-region replication (optional)
    \item Lifecycle policies for archival
\end{itemize}

\subsection{Infrastructure as Code}
\begin{itemize}
    \item Recommended: Terraform or AWS CDK
    \item Version-controlled infrastructure
    \item Reproducible deployments
    \item Environment parity (dev/staging/prod)
\end{itemize}

\section{Conclusion}

This document outlines a comprehensive AWS deployment strategy for the Kwalifai Mortgage Platform. The recommended architecture provides:

\begin{itemize}
    \item \textbf{Scalability}: Auto-scaling capabilities for all services
    \item \textbf{Reliability}: Multi-AZ deployment, automated backups
    \item \textbf{Security}: Network isolation, encryption, WAF protection
    \item \textbf{Cost-Effectiveness}: Estimated \$260/month for startup scale
    \item \textbf{Maintainability}: Managed services reduce operational overhead
    \item \textbf{Future-Proof}: Architecture supports growth and additional features
\end{itemize}

The selected AWS services provide the optimal balance of features, cost, and operational simplicity for a microservices-based mortgage platform. The architecture can scale from startup to enterprise levels while maintaining cost efficiency and operational excellence.

\section{Appendices}

\subsection{Appendix A: Service Ports}
\begin{itemize}
    \item Frontend: 3000
    \item API Gateway: 3001
    \item Auth Service: 3002
    \item Calculator Service: 3003
    \item Rate Alert Service: 3004
    \item Statement Analysis Service: 3005
    \item AI Service: 3006
\end{itemize}

\subsection{Appendix B: Environment Variables}
Key environment variables needed for deployment:
\begin{itemize}
    \item \texttt{DATABASE\_URL}
    \item \texttt{OPENAI\_API\_KEY}
    \item \texttt{JWT\_SECRET}
    \item \texttt{AWS\_S3\_BUCKET}
    \item \texttt{AWS\_SES\_REGION}
    \item Service-specific ports and URLs
\end{itemize}

\subsection{Appendix C: Useful AWS CLI Commands}
\begin{lstlisting}
# Check ECS service status
aws ecs describe-services --cluster my-cluster --services my-service

# View CloudWatch logs
aws logs tail /ecs/my-service --follow

# Check RDS instance status
aws rds describe-db-instances --db-instance-identifier my-db

# List S3 objects
aws s3 ls s3://my-bucket/ --recursive
\end{lstlisting}

\end{document}
